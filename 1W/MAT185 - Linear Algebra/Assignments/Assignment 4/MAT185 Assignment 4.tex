\documentclass[10pt]{exam}

\usepackage{mathtools}
\usepackage{amsmath}
\usepackage{amssymb}
\usepackage{color}
\usepackage{graphicx}
\usepackage{pgfplots}
\usepackage[margin=0.6in]{geometry}
\usepackage{tikz}
\usepackage{float}
\usepackage[hidelinks, urlcolor=blue, linkcolor=blue, colorlinks=true]{hyperref} 
\usepackage{tcolorbox}

\DeclarePairedDelimiterX\set[1]\lbrace\rbrace{\def\given{\;\delimsize\vert\;}#1}

\newcommand{\bcent}{\begin{center}}
\newcommand{\ecent}{\end{center}}
\newcommand{\tb}{\textbf}
\newcommand{\noin}{\noindent}
\newcommand{\benum}{\begin{enumerate}}
\newcommand{\eenum}{\end{enumerate}}
\newcommand{\bitem}{\begin{itemize}}
\newcommand{\eitem}{\end{itemize}}
%%% This command makes a framed box of a chosen height.
\newcommand{\makenonemptybox}[2]{%
\par\nobreak\vspace{\ht\strutbox}\noindent
\setlength{\fboxrule}{0pt} % set this to 0pt to make invisible
\fbox{%
\parbox[c][#1][t]{\dimexpr\linewidth-2\fboxsep}{
  \hrule width \hsize height 0pt
  #2
 }%
}%
}
\makeatother

    

\begin{document}

{\bcent\fontfamily{cmss}\selectfont
\begin{tabular}{c}
\textbf{}~~~~~~~~~~~~~~~~~~~~~~~~~~~~~~~~~~~~~~~~~~~~~~~~~~~~~~~~~~~~~~~~~~~~~~~~~~~~~~~~~~~~~~~\textbf{Due: 11:59pm ET, Friday March 19, 2021}\\\hline
\end{tabular}\ecent
}

{\fontfamily{cmss}\selectfont
\large\bcent\tb{}\\
\tb{}\\
\vspace{0pt}

\tb{\Large MAT185 Linear Algebra}\\

\tb{Assignment 4}
\ecent}



\noin{\fontfamily{cmss}\selectfont\tb{\large Instructions:}} \\ %% Fairly standard and designed to save time; however, tweak as necessary.

\noindent Please read the {\color{red}Assignment Policies, FAQ, and Rubric} document for details on submission policies, collaboration rules and academic integrity, and general instructions. 

\benum


\item {\bf Submissions are only accepted by} \href{https://www.gradescope.ca}{Gradescope}. Do not send anything by email.  Late submissions are not accepted under any circumstance. Remember you can resubmit anytime before the deadline. 

\item  {\bf Submit your polished solutions using only this template pdf}.  Your submission should be a single pdf with your full written solutions for each question. If your solution is not written using this template pdf (scanned print or digital) then your submission will not be assessed. Organize your work neatly in the space provided.  Do not submit rough work. 

\item  {\bf Show your work and justify your steps} on every question but do not include extraneous information.  Put your final answer in the box provided, if necessary.  We recommend you write draft solutions on separate pages and afterwards write your polished solutions here on this template.

\item  {\bf You must fill out and sign the academic integrity statement below}; otherwise, you will receive zero for this assignment. 


\eenum

\vspace{30pt}


\noin{\fontfamily{cmss}\selectfont\tb{\large Academic Integrity Statement:}} \\

%%% Student information

% Student 1
\fbox{
\begin{minipage}{\textwidth}
{
\vspace{0.2in}

\makebox[\textwidth]{\sffamily Full Name:\enspace Regis Zhao \hrulefill}

\vspace{0.2in}

\makebox[\textwidth]{\sffamily Student number:\enspace 1007070660 \hrulefill}

\vspace{0.1in}
}
\end{minipage}
}

\vspace*{0.1in}

% Student 2
\fbox{
\begin{minipage}{\textwidth}
{
\vspace{0.2in}

\makebox[\textwidth]{\sffamily Full Name:\enspace Sherry Zhang \hrulefill}

\vspace{0.2in}

\makebox[\textwidth]{\sffamily Student number:\enspace 1007145297 \hrulefill}

\vspace{0.1in}
}
\end{minipage}
}
~

I confirm that:

\begin{itemize} 
	\item I have read and followed the policies described in the document {\color{red} Assignment Policies, FAQ, and Rubric}.
	\item In particular, I have read and understand the rules for collaboration, and permitted resources on assignments as described in subsection II of the the aforementioned document. I have not violated these rules while completing and writing this assignment. 
	\item I understand the consequences of violating the University's academic integrity policies as outlined in the \href{http://www.governingcouncil.utoronto.ca/policies/behaveac.htm}{Code of Behaviour on Academic Matters}. I have not violated them while completing and writing this assignment.
\end{itemize}
By signing this document, I agree that the statements above are true. 

% You should sign this PDF after compiling. Do not write your signature using LaTeX.
\vspace{0.2in}
{\large 
\makebox[\textwidth]{\sffamily Signatures: 1)\enspace\hrulefill} 

\vspace{0.2in}

\makebox[\textwidth]{\sffamily \hspace*{20mm} 2)\enspace\hrulefill} 

}

\vfill


\pagebreak

%%% Questions

\noin {\tb 1.}   Let $\alpha$ be a basis for an $n$-dimensional vector space $V$, and let ${\bf v}_1, {\bf v}_2, \dots {\bf v}_k \in V$.  Prove that $\mathrm{span}\, \{ {\bf v}_1, {\bf v}_2, \dots {\bf v}_k \}=V$ if and only if $\mathrm{span}\, \{ [{\bf v}_1]_{\alpha}, [{\bf v}_2]_{\alpha}, \dots, [{\bf v}_k]_{\alpha} \} = {^n}\mathbb R$, where $[{\bf v}_j]_{\alpha}$ denotes the coordinate vector of ${\bf v}_j$ for each $j=1,2,\dots,k.$


		\fullwidth{
			\vspace*{-10pt}
			%%% Do not change the height of this box. Your work must fit inside it.
			\makenonemptybox{550pt}{ 


			%%% Your work goes here!

			First we show that if $\mathrm{span}\, \{ {\bf v}_1, {\bf v}_2, \dots {\bf v}_k \}=V$, then $\mathrm{span}\, \{ [{\bf v}_1]_{\alpha}, [{\bf v}_2]_{\alpha}, \dots, [{\bf v}_k]_{\alpha} \} = {^n}\mathbb R$.\\
			
			If $\mathrm{span}\, \{ {\bf v}_1, {\bf v}_2, \dots {\bf v}_k \}=V$, then there is a sublist of this spanning set that is a basis for $V$ (by Medici 6.4 Theorem VII) which contains $n$ vectors: $\{ {\bf v}_1, {\bf v}_2, \dots {\bf v}_n \}$. Since $\{ {\bf v}_1, {\bf v}_2, \dots {\bf v}_n \}$ is a basis, it must be linearly independent. Then, by Medici 8.4 Theorem I, we have that the corresponding set of $n$ coordinates $\{ [{\bf v}_1]_{\alpha}, [{\bf v}_2]_{\alpha}, \dots, [{\bf v}_n]_{\alpha} \}$ are linearly independent in ${^n}\mathbb R$, and so $\{ [{\bf v}_1]_{\alpha}, [{\bf v}_2]_{\alpha}, \dots, [{\bf v}_n]_{\alpha} \}$ must be a basis for ${^n}\mathbb R$, i.e. $\mathrm{span}\, \{ [{\bf v}_1]_{\alpha}, [{\bf v}_2]_{\alpha}, \dots, [{\bf v}_n]_{\alpha} \} = {^n}\mathbb R$. Since $\{ [{\bf v}_1]_{\alpha}, [{\bf v}_2]_{\alpha}, \dots, [{\bf v}_n]_{\alpha} \}$ is a basis for ${^n}\mathbb R$, any vector $\textbf{v} \in {^n}\mathbb R$ added to that set will make the set linearly dependent and will not affect the span of the set. Therefore, $\mathrm{span}\, \{ [{\bf v}_1]_{\alpha}, [{\bf v}_2]_{\alpha}, \dots, [{\bf v}_k]_{\alpha} \} = {^n}\mathbb R$. This process is summarized below:
			\begin{gather*}
				\mathrm{span}\, \{ {\bf v}_1, {\bf v}_2, \dots {\bf v}_k \} = V\\
				\mathrm{span}\, \{ {\bf v}_1, {\bf v}_2, \dots {\bf v}_k \} = \mathrm{span}\, \{ {\bf v}_1, {\bf v}_2, \dots {\bf v}_n \} \quad\quad n\leq k \quad\quad \text{(by Medici 6.4 Theorem VII)}\\
				\implies \mathrm{span}\, \{ [{\bf v}_1]_{\alpha}, [{\bf v}_2]_{\alpha}, \dots, [{\bf v}_n]_{\alpha} \} = {^n}\mathbb R \quad\quad \text{(by Medici 8.4 Theorem I)}\\
				\therefore \mathrm{span}\, \{ [{\bf v}_1]_{\alpha}, [{\bf v}_2]_{\alpha}, \dots, [{\bf v}_k]_{\alpha} \} = {^n}\mathbb R \quad\quad \text{(by Medici 6.4 Theorem VII)}
			\end{gather*}\\

			Now we show the other direction: if $\mathrm{span}\, \{ [{\bf v}_1]_{\alpha}, [{\bf v}_2]_{\alpha}, \dots, [{\bf v}_k]_{\alpha} \} = {^n}\mathbb R$, then $\mathrm{span}\, \{ {\bf v}_1, {\bf v}_2, \dots {\bf v}_k \}=V$. The proof is essentially the reverse of the first direction:
			\begin{gather*}
				\mathrm{span}\, \{ [{\bf v}_1]_{\alpha}, [{\bf v}_2]_{\alpha}, \dots, [{\bf v}_k]_{\alpha} \} = {^n}\mathbb R\\
				\mathrm{span}\, \{ [{\bf v}_1]_{\alpha}, [{\bf v}_2]_{\alpha}, \dots, [{\bf v}_k]_{\alpha} \} = \mathrm{span}\, \{ [{\bf v}_1]_{\alpha}, [{\bf v}_2]_{\alpha}, \dots, [{\bf v}_n]_{\alpha} \} \quad\quad n\leq k \quad\quad \text{(by Medici 6.4 Theorem VII)}\\
				\implies \mathrm{span}\, \{ {\bf v}_1, {\bf v}_2, \dots {\bf v}_n \} = V \quad\quad \text{(by Medici 8.4 Theorem I)}\\
				\therefore \mathrm{span}\, \{ {\bf v}_1, {\bf v}_2, \dots {\bf v}_k \} = V \quad\quad \text{(by Medici 6.4 Theorem VII)}
			\end{gather*}\\

			Therefore, $\mathrm{span}\, \{ {\bf v}_1, {\bf v}_2, \dots {\bf v}_k \}=V$ if and only if $\mathrm{span}\, \{ [{\bf v}_1]_{\alpha}, [{\bf v}_2]_{\alpha}, \dots, [{\bf v}_k]_{\alpha} \} = {^n}\mathbb R$.


			}
		}
	





\pagebreak 

	% Question 2	

\noin {\bf 2.}   Let ${\bf u}=(u_1, u_2)\, {\bf v}=(v_1, v_2)$ be vectors in $\mathbb R^2$.  Prove that the area of the parallelogram with vertices ${\bf 0}$, ${\bf u}$, ${\bf v}$, and ${\bf u}+{\bf v}$ is
$$| \det \begin{bmatrix} {\bf u} \\ {\bf v} \end{bmatrix} |$$

\noin where $| \ \ \ |$ denotes the absolute value.




		\fullwidth{
			\vspace*{-10pt}
			%%% Do not change the height of this box. Your work must fit inside it.
			\makenonemptybox{550pt}{ 


			%%% Your work goes here!

			Even though $\textbf{u}$ and $\textbf{v}$ are vectors in $\mathbb R^2$, we can think of them geometrically as vectors in $\mathbb R^3$ (3D space) that lie within the same 2D plane. We can represent vectors $\textbf{u}$ and $\textbf{v}$ in $\mathbb R^3$ as $\textbf{u}' = (u_1, u_2, 0)$ and $\textbf{v}' = (v_1, v_2, 0)$.\\

			By Stewart Calculus - Section 12.4, the area of a parallelogram with its edges defined by two vectors can be determined by taking the magnitude of the cross product of the vectors. Given two vectors $\textbf{a} = (a_1, a_2, a_3)$ and $\textbf{b} = (b_1, b_2, b_3)$, their cross product is defined as: $\textbf{a}\times \textbf{b} = (a_2b_3-a_3b_2, \ a_3b_1-a_1b_3,\ a_1b_2-a_2b_1)$. Since the edges of the parallelogram described in the question are defined by vectors $\textbf{u}$ and $\textbf{v}$, we take the cross product of $\textbf{u}'$ and $\textbf{v}'$:
			\begin{align*}
				\textbf{u}' \times \textbf{v}' &= (u_2(0)-(0)v_2, \ (0)v_1-u_1(0),\ u_1v_2-u_2v_1)\\
				&= (0, \ 0, \ u_1v_2-u_2v_1)
			\end{align*}\\

			The magnitude of a vector $\textbf{a} = (a_1, a_2, a_3) \in \mathbb R^3$ is given by: $|\textbf{a}| = \sqrt{a_1^2+a_2^2+a_3^2}$ (Stewart Calculus - Section 12.2). We take the magnitude of the cross product $\textbf{u}' \times \textbf{v}'$ to find the area of the parallelogram:
			\begin{align*}
				\text{area of parallelogram} &= |\textbf{u}' \times \textbf{v}'|\\
				&= \sqrt{0^2 + 0^2 + (u_1v_2-u_2v_1)^2}\\
				&= \sqrt{(u_1v_2-u_2v_1)^2}\\
				&= |u_1v_2-u_2v_1|
			\end{align*}

			Now we evaluate $| \det \begin{bmatrix} {\bf u} \\ {\bf v} \end{bmatrix} |$:
			\begin{align*}
				| \det \begin{bmatrix} {\bf u} \\ {\bf v} \end{bmatrix} | &= | \det \begin{bmatrix} u_1 & u_2 \\ v_1 & v_2 \end{bmatrix} |\\
				&= |u_1v_2 - u_2v_1| \quad\quad \text{(by Medici 9.1 Eq. 1)}
			\end{align*}
			\begin{displaymath}
				\therefore \text{area of parallelogram} = | \det \begin{bmatrix} {\bf u} \\ {\bf v} \end{bmatrix} |
			\end{displaymath}
			}
		}



\end{document}
