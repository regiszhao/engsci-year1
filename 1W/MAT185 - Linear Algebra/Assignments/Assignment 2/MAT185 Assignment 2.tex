\documentclass[10pt]{exam}

\usepackage{mathtools}
\usepackage{amsmath}
\usepackage{amssymb}
\usepackage{color}
\usepackage{graphicx}
\usepackage[margin=0.6in]{geometry}
\usepackage{tikz}
\usepackage{float}
\usepackage[hidelinks, urlcolor=blue, linkcolor=blue, colorlinks=true]{hyperref} 
\usepackage{tcolorbox}

\DeclarePairedDelimiterX\set[1]\lbrace\rbrace{\def\given{\;\delimsize\vert\;}#1}

\newcommand{\bcent}{\begin{center}}
\newcommand{\ecent}{\end{center}}
\newcommand{\tb}{\textbf}
\newcommand{\noin}{\noindent}
\newcommand{\benum}{\begin{enumerate}}
\newcommand{\eenum}{\end{enumerate}}
\newcommand{\bitem}{\begin{itemize}}
\newcommand{\eitem}{\end{itemize}}
%%% This command makes a framed box of a chosen height.
\newcommand{\makenonemptybox}[2]{%
\par\nobreak\vspace{\ht\strutbox}\noindent
\setlength{\fboxrule}{0pt} % set this to 0pt to make invisible
\fbox{%
\parbox[c][#1][t]{\dimexpr\linewidth-2\fboxsep}{
  \hrule width \hsize height 0pt
  #2
 }%
}%
}
\makeatother

    

\begin{document}

{\bcent\fontfamily{cmss}\selectfont
\begin{tabular}{c}
\textbf{}~~~~~~~~~~~~~~~~~~~~~~~~~~~~~~~~~~~~~~~~~~~~~~~~~~~~~~~~~~~~~~~~~~~~~~~~~~~~~~~~~~~~~~~\textbf{Due: 11:59pm ET, Friday February 5, 2021}\\\hline
\end{tabular}\ecent
}

{\fontfamily{cmss}\selectfont
\large\bcent\tb{}\\
\tb{}\\
\vspace{0pt}

\tb{\Large MAT185 Linear Algebra}\\

\tb{Assignment 2}
\ecent}



\noin{\fontfamily{cmss}\selectfont\tb{\large Instructions:}} \\ %% Fairly standard and designed to save time; however, tweak as necessary.

\noindent Please read the {\color{red}Assignment Policies, FAQ, and Rubric} document for details on submission policies, collaboration rules and academic integrity, and general instructions. 

\benum


\item {\bf Submissions are only accepted by} \href{https://www.gradescope.ca}{Gradescope}. Do not send anything by email.  Late submissions are not accepted under any circumstance. Remember you can resubmit anytime before the deadline. 

\item  {\bf Submit your polished solutions using only this template pdf}.  Your submission should be a single pdf with your full written solutions for each question. If your solution is not written using this template pdf (scanned print or digital) then your submission will not be assessed. Organize your work neatly in the space provided.  Do not submit rough work. 

\item  {\bf Show your work and justify your steps} on every question but do not include extraneous information.  Put your final answer in the box provided, if necessary.  We recommend you write draft solutions on separate pages and afterwards write your polished solutions here on this template.

\item  {\bf You must fill out and sign the academic integrity statement below}; otherwise, you will receive zero for this assignment. 


\eenum

\vspace{30pt}


\noin{\fontfamily{cmss}\selectfont\tb{\large Academic Integrity Statement:}} \\

%%% Student information

% Student 1
\fbox{
\begin{minipage}{\textwidth}
{
\vspace{0.2in}

\makebox[\textwidth]{\sffamily Full Name: Regis Zhao\enspace\hrulefill}

\vspace{0.2in}

\makebox[\textwidth]{\sffamily Student number: 1007070660\enspace\hrulefill}

\vspace{0.1in}
}
\end{minipage}
}

\vspace*{0.1in}

% Student 2
\fbox{
\begin{minipage}{\textwidth}
{
\vspace{0.2in}

\makebox[\textwidth]{\sffamily Full Name: Sherry Zhang\enspace\hrulefill}

\vspace{0.2in}

\makebox[\textwidth]{\sffamily Student number: 1007145297\enspace\hrulefill}

\vspace{0.1in}
}
\end{minipage}
}
~

I confirm that:

\begin{itemize} 
	\item I have read and followed the policies described in the document {\color{red} Assignment Policies, FAQ, and Rubric}.
	\item In particular, I have read and understand the rules for collaboration, and permitted resources on assignments as described in subsection II of the the aforementioned document. I have not violated these rules while completing and writing this assignment. 
	\item I understand the consequences of violating the University's academic integrity policies as outlined in the \href{http://www.governingcouncil.utoronto.ca/policies/behaveac.htm}{Code of Behaviour on Academic Matters}. I have not violated them while completing and writing this assignment.
\end{itemize}
By signing this document, I agree that the statements above are true. 

% You should sign this PDF after compiling. Do not write your signature using LaTeX.
\vspace{0.2in}
{\large 
\makebox[\textwidth]{\sffamily Signatures: 1)\enspace\hrulefill} 

\vspace{0.2in}

\makebox[\textwidth]{\sffamily \hspace*{20mm} 2)\enspace\hrulefill} 

}

\vfill


\pagebreak

%%% Questions

\noin {\tb 1.}  Consider the vector space $\mathbb R^3 = \{  {\bf x} \mid {\bf x}=(x_1, x_2, x_3),\,\,  x_1, x_2, x_3 \in \mathbb R \}$, with respect to the usual vector addition and scalar multiplication.  For each ${\bf x}, {\bf y}\in \mathbb R^3$, define the dot product as
$${\bf x}\cdot {\bf y}= x_1y_1+x_2y_2+x_3y_3.$$

\vspace{20pt}

	% Question 1(a)	
	
\noin {\bf 1(a)}   For any fixed ${\bf n} \in \mathbb R^3$, show that the set $S=\{ {\bf x}\in \mathbb R^3 \mid {\bf n}\cdot {\bf x} =0 \}$ is a subspace of $\mathbb R^3$.
\\
\\
\\
A theorem we learned in class states that a non-empty subset $W$ of a vector 
space $V$ is a subspace of $V$ \emph{iff} $c\textbf{x}+\textbf{y} \in W$, whenever
$\textbf{x}, \textbf{y} \in W$ and $c \in \mathbb{R}$. 
\\
\\
Therefore, we need to show that $c\textbf{x}+\textbf{y} \in S$ when $\textbf{x}, 
\textbf{y} \in S$ and $c \in \mathbb{R}$:
\begin{align*}
	\textbf{n} \cdot (c\textbf{x}+\textbf{y}) &= (n_1, n_2, n_3) \cdot (c(x_1,x_2,x_3)+(y_1,y_2,y_3)) \\
	&= (n_1, n_2, n_3) \cdot (cx_1+y_1, cx_2+y_2, cx_3+y_3) \\
	&= n_1(cx_1+y_1) + n_2(cx_2+y_2) + n_3(cx_3+y_3) \\
	&= cn_1x_1 + n_1y_1 + cn_2x_2 + n_2y_2 + cn_3x_3 + n_3y_3 \\
	&= c(n_1x_1 + n_2x_2 + n_3x_3) + (n_1y_1 + n_2y_2 + n_3y_3) \\
	&= c(0) + 0 \\
	&= 0 \\
	\therefore c\textbf{x}+\textbf{y} \in S
\end{align*}
Therefore, the subset $S$ is a subspace of $\mathbb{R}^3$.

\vspace{100pt}		%%%Do not change this spacing.



	% Question 1(b)	

\noin {\bf 1(b)} Describe the set $S$ in part (a) geometrically.
\\
\\
\\
Given that both vectors aren't the zero vector, if the dot product between 
two vectors is 0, then the two vectors are perpendicular. Therefore, 
geometrically, the set $S$ can be represented as all vectors that are perpendicular 
to $\textbf{n}$ in 3D space, essentially forming a plane that is perpendicular to 
$\textbf{n}$.		%%% Your answer goes here!


\pagebreak 

	% Question 2	

\noin {\bf 2.}   The set
$$U=\{ A \in \,^3\mathbb R^3 \mid A=A^T \mbox { and } \mathrm{tr}\, A = 0 \}$$
is a subspace of $^3\mathbb R^3$ (with respect to the usual matrix addition and scalar multiplication).  Find a finite subset $S$ of $^3\mathbb R^3$ such that $\mathrm{span}\, S =U$.\\

\noin Recall that $\mathrm{tr}\, A$ denotes the {\it trace} of $A$, defined as the sum of the diagonal entries of $A$.
\\
\\
\\
The condition that $A = A^T$ means that $A$ must be symmetrical about the 
diagonal. The following set of matrices account for this symmetry (any linear 
combination of these matrices will produce a matrix $A$ such that $A = A^T$):

\begin{center}
$\bigg\{ \begin{pmatrix}
	0 & 1 & 0 \\
	1 & 0 & 0 \\
	0 & 0 & 0 
\end{pmatrix},
\begin{pmatrix}
	0 & 0 & 1 \\
	0 & 0 & 0 \\
	1 & 0 & 0 
\end{pmatrix},
\begin{pmatrix}
	0 & 0 & 0 \\
	0 & 0 & 1 \\
	0 & 1 & 0 
\end{pmatrix}\bigg\}$.
\end{center}
Now we need to account for all possible matrices where the sum of the diagonal entries equal 
to 0 (note that all diagonal entries of the matrices above are 0 -- they don't affect this condition). 
We can do this by ensuring that each diagonal entry has a matrix in the 
subset where that entry has a non-zero value (e.g. 1), and that value's negative (e.g. -1)
is placed somewhere else along the diagonal (so that that the sum of diagonal 
entries is always 0). All other entries not on the diagonal will be set to 0 
so that they do not affect the symmetry of the matrix. For example:

\begin{center}
$\bigg\{ \begin{pmatrix}
	1 & 0 & 0 \\
	0 & -1 & 0 \\
	0 & 0 & 0 
\end{pmatrix},
\begin{pmatrix}
	1 & 0 & 0 \\
	0 & 0 & 0 \\
	0 & 0 & -1 
\end{pmatrix}\bigg\}$.
\end{center}
In the set shown above, each diagonal entry spot of a 3-by-3 matrix has a matrix where 
that entry is non-zero and has its negative value placed elsewhere along the diagonal.
\\\\
Combining these two sets, we get a finite subset $S$ of $^3\mathbb{R}^3$ such that 
for all $A \in U$, $A$ can be represented as a linear combination of the elements 
in $S$, i.e. $\mathrm{span}\, S =U$:

\begin{center}
$S = \bigg\{ \begin{pmatrix}
	0 & 1 & 0 \\
	1 & 0 & 0 \\
	0 & 0 & 0 
\end{pmatrix},
\begin{pmatrix}
	0 & 0 & 1 \\
	0 & 0 & 0 \\
	1 & 0 & 0 
\end{pmatrix},
\begin{pmatrix}
	0 & 0 & 0 \\
	0 & 0 & 1 \\
	0 & 1 & 0 
\end{pmatrix}
\begin{pmatrix}
	1 & 0 & 0 \\
	0 & -1 & 0 \\
	0 & 0 & 0 
\end{pmatrix},
\begin{pmatrix}
	1 & 0 & 0 \\
	0 & 0 & 0 \\
	0 & 0 & -1 
\end{pmatrix}\bigg\}$.
\end{center}


		%%% Your answer goes here!


		%%% Do not add extra pages.




\end{document}
