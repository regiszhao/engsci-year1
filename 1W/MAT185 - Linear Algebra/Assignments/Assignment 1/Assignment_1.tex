\documentclass[10pt, letterpaper]{article}
\usepackage[utf8]{inputenc}
\usepackage{amsmath}
\usepackage{amssymb}
\usepackage[margin=0.75in]{geometry}
\makeatletter
\newcommand*{\rom}[1]{\expandafter\@slowromancap\romannumeral #1@}
\makeatother

\title{Assignment 1}
\author{Regis Zhao 1007070660}
\date{\today}

\begin{document}

\maketitle

% QUESTION 1

\section{Question 1}

% PART 1A
\subsection{Part a}
Axioms ii and ix from the "Peano system" are not present in the definition
of a vector space.

% PART 1B
\subsection{Part b}
Axioms A\rom{3} and A\rom{4} in Medici's definition of a vector space are not
present in the "Peano system".

% PART 1C
\subsection{Part c}
Since the only axioms that are not included in the definition of a "Peano system"
are A\rom{3} and A\rom{4}, we must prove that a "Peano system" satisfies
these two axioms.

% Proof for AIII
\textbf{A\rom{3})} Prove there exists a zero vector $\textbf{0} \in V$ such that 
$ \textbf{x} + \textbf{0} = \textbf{x} $:
%
\begin{align*}
    \textbf{x} + \textbf{0} &= \textbf{x} + 0\textbf{x} \qquad \text{by axiom ix}\\
    &= 1\textbf{x} + 0\textbf{x} \qquad \text{by axiom viii}\\
    &= (1+0)\textbf{x} \qquad \text{by axiom vi}\\
    &= 1\textbf{x} \qquad \text{by arithmetic}\\
    &= \textbf{x} \qquad \text{by axiom viii}
\end{align*}
$\therefore$ Axiom A\rom{3} holds.

%Proof for AIV
\textbf{A\rom{4})} Prove there exists a negative $\textbf{-x} \in V$ such that 
$ \textbf{x} + (\textbf{-x}) = \textbf{0} $:
%
\begin{align*}
    \textbf{0} &= 0\textbf{x} \qquad \text{by axiom ix}\\
    &= (1+(-1))\textbf{x} \qquad \text{by arithmetic}\\
    &= 1\textbf{x} + (-1) \textbf{x} \qquad \text{by axiom vi}\\
    &= \textbf{x} + (-1) \textbf{x} \qquad \text{by axiom viii}
\end{align*}
%
We have shown that there exists a vector $\textbf{-x} \in V$ such 
that $ \textbf{x} + (\textbf{-x}) = \textbf{0} $, where 
$\textbf{-x} = (-1) \textbf{x} $. $\therefore$ Axiom A\rom{4} holds.

%------------------------------------------------------------------%

% QUESTION 2

\section{Question 2}

The given differential equation (DE) that the set of all $x(t)$ must satisfy is a second order homogeneous linear DE. 
From calculus, we know that if two linearly independent functions, say $f$ and $g$, are both 
solutions to such an equation, then the general solution is any linear 
combination of $f$ and $g$, i.e. $x(t) = c_1f + c_2g, \text{where scalars} \ c_1, c_2 \in \mathbb{R} $.

We prove that all 8 axioms that define a vector space hold for this set of all $x$:

Let
\begin{gather*}
    \textbf{u} = a_1f + a_2g \in V \\
    \textbf{v} = b_1f + b_2g \in V \\
    \textbf{w} = c_1f + c_2g \in V
\end{gather*}
where scalars $a_1, a_2, b_1, b_2, c_1, c_2 \in \mathbb{R}$.

% Proof for AI
\textbf{A\rom{1})} \textsc{Additive Closure}
\begin{align*}
    \textbf{u} + \textbf{v} &= (a_1f + a_2g) + (b_1f + b_2g) \\
    &= a_1f + a_2g + b_1f + b_2g \\
    &= (a_1 + b_1)f + (a_2 + b_2)g 
\end{align*}
Since $a_1 + b_1, a_2 + b_2 \in \mathbb{R}$:
\begin{gather*}
    (a_1 + b_1)f + (a_2 + b_2)g \in V \\
    \therefore \textbf{u} + \textbf{v} \in V
\end{gather*}

% Proof for AII
\textbf{A\rom{2})} \textsc{Additive Associativity} \\
We show that $ ( \textbf{u} + \textbf{v} ) + \textbf{w} = \textbf{u} + (\textbf{v} + \textbf{w})$
by evaluating the left and right hand side of the equation.\\
The left hand side gives:
\begin{align*}
    ( \textbf{u} + \textbf{v} ) + \textbf{w} &= ((a_1f + a_2g) + (b_1f + b_2g)) + (c_1f + c_2g) \\
    &= (a_1f + a_2g) + (b_1f + b_2g) + (c_1f + c_2g)
\end{align*}
The right hand side gives:
\begin{align*}
    \textbf{u} + (\textbf{v} + \textbf{w}) &= (a_1f + a_2g) + ((b_1f + b_2g) + (c_1f + c_2g)) \\
    &= (a_1f + a_2g) + (b_1f + b_2g) + (c_1f + c_2g)
\end{align*}
\begin{displaymath}
    \therefore ( \textbf{u} + \textbf{v} ) + \textbf{w} = \textbf{u} + (\textbf{v} + \textbf{w})
\end{displaymath}

% Proof for AIII
\textbf{A\rom{3})} \textsc{Zero Vector} \\
Let $ \textbf{0} = z_1f + z_2g \in V $ be an arbitrary vector. 
We pretend it satisfies the axiom $\textbf{u} + \textbf{0} = \textbf{u}$ and 
see if a solution exists:
\begin{align*}
    \textbf{u} + \textbf{0} &= \textbf{u} \\
    (a_1f + a_2g) + (z_1f + z_2g) &= (a_1f + a_2g) \\
    a_1f + a_2g + z_1f + z_2g &= a_1f + a_2g \\
    (a_1 + z_1)f + (a_2 + z_2)g &= a_1f + a_2g \\
\end{align*}
It follows that
\begin{gather*}
    a_1 + z_1 = a_1 \\
    \therefore z_1 = 0
\end{gather*}
and
\begin{gather*}
    a_2 + z_2 = a_2 \\
    \therefore z_2 = 0
\end{gather*}
Therefore the zero vector exists, and $ \textbf{0} = 0f + 0g \in V $.

% Proof for AIII
\textbf{A\rom{4})} \textsc{Additive Inverse} \\
Let $ \textbf{-u} = n_1f + n_2g \in V $ be an arbitrary vector. 
We pretend it satisfies the axiom $\textbf{u} + (\textbf{-u}) = \textbf{0}$ and 
see if a solution exists:
\begin{align*}
    \textbf{u} + (\textbf{-u}) &= \textbf{u} \\
    (a_1f + a_2g) + (n_1f + n_2g) &= 0f + 0g \\
    a_1f + a_2g + n_1f + n_2g &= 0f + 0g \\
    (a_1 + n_1)f + (a_2 + n_2)g &= 0f + 0g \\
\end{align*}
It follows that
\begin{gather*}
    a_1 + n_1 = 0 \\
    \therefore n_1 = -a_1
\end{gather*}
and
\begin{gather*}
    a_2 + n_2 = 0 \\
    \therefore n_2 = -a_2
\end{gather*}
Therefore the additive inverse of a vector $\textbf{u}$ exists, and $ \textbf{-u} = -a_1f + -a_2g \in V $.

% Proof for MI
\textbf{M\rom{1})} \textsc{Scalar Closure} \\
Let $\alpha \in \mathbb{R}$ be a scalar:
\begin{align*}
    \alpha \textbf{u} &= \alpha(a_1f + a_2g) \\
    &= \alpha a_1f + \alpha a_2g  \\
\end{align*}
Since $\alpha a_1, \alpha a_2 \in \mathbb{R}$:
\begin{gather*}
    \alpha a_1f + \alpha a_2g \in V \\
    \therefore \alpha \textbf{u} \in V
\end{gather*}

% Proof of MII
\textbf{M\rom{2})} \textsc{Scalar Associativity} \\
Let $\beta \in \mathbb{R}$ be a scalar:
\begin{align*}
    \alpha (\beta \textbf{u}) &= \alpha(\beta a_1f + \beta a_2g) \\
    &= \alpha \beta a_1f + \alpha \beta a_2g
\end{align*}
Also:
\begin{align*}
    (\alpha \beta) \textbf{u} &= \alpha \beta (a_1f + a_2g) \\
    &= \alpha \beta a_1f + \alpha \beta a_2g
\end{align*}
\begin{displaymath}
    \therefore \alpha (\beta \textbf{u}) = (\alpha \beta) \textbf{u}
\end{displaymath}

% Proof of MII
\textbf{M\rom{3})} \textsc{Distributivity} \\
Distributivity of scalar addition:
\begin{align*}
    (\alpha+\beta) \textbf{u} &= (\alpha+\beta)(a_1f + a_2g) \\
    &= \alpha a_1f + \alpha a_2g + \beta a_1f + \beta a_2g \\
    &= \alpha(a_1f + a_2g) + \beta(a_1f + a_2g) \\
    &= \alpha \textbf{u} + \beta \textbf{u}
\end{align*}
Distributivity of vector addition:
\begin{align*}
    \alpha(\textbf{u} + \textbf{v}) &= \alpha((a_1f + a_2g) + (b_1f + b_2g)) \\
    &= \alpha(a_1f + a_2g + b_1f + b_2g) \\
    &= \alpha a_1f + \alpha a_2g + \alpha b_1f + \alpha b_2g \\
    &= \alpha(a_1f + a_2g) + \alpha(b_1f + b_2g) \\
    &= \alpha \textbf{u} + \alpha \textbf{v}
\end{align*}

% Proof of MII
\textbf{M\rom{4})} \textsc{Identity} \\
\begin{align*}
    1\textbf{u} &= 1(a_1f + a_2g) \\
    &= 1a_1f + 1a_2g \\
    &= a_1f + a_2g \\
    &= \textbf{u}
\end{align*}
\begin{displaymath}
    \therefore 1\textbf{u} = \textbf{u}
\end{displaymath}

Sherry Zhang 1007145297

\end{document}